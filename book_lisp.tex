\documentclass[11pt]{book}              % Book class in 11 points
\parindent0pt  \parskip10pt             % make block paragraphs
\raggedright                            % do not right justify

\title{\bf Notas sobre el Lenguaje Lisp}    % Supply information
\author{[Lenguajes de Prog. ECFM/UAP/HJS]}              %   for the title page.
\date{\today}                           %   Use current date. 

% Note that book class by default is formatted to be printed back-to-back.
\begin{document}                        % End of preamble, start of text.
\frontmatter                            % only in book class (roman page #s)
\maketitle                              % Print title page.
\tableofcontents                        % Print table of contents
\mainmatter                             % only in book class (arabic page #s)
\part{Introducci\'on}                   % Print a "part" heading
\chapter{Introducci\'on}                % Print a "chapter" heading

El lenguaje de Programaci\'on  LISP esta orientado al manejo de listas mediante la evaluaci\'on de funciones. Por tal motivo se dice que es un lenguaje funcional. Toda  operaci\'on en LISP se expresa mediante la composici\'on de funciones.\\
La orientaci\'on  de LISP al manejo de listas, se basa en que la lista es una estructura de datos muy versatil. Por ejemplo mediante listas es posible representar  \'arboles. De esta manera podemos organizar informaci\'on compleja jeraquicamente. Por otro lado, las operaciones  del lenguaje se inclinan al manejo simb\'olico.\\


Most of this example applies to \texttt{article} and \texttt{book} classes
as well as to \texttt{report} class. In \texttt{article} class, however,
the default position for the title information is at the top of
the first text page rather than on a separate page. Also, it is
not usual to request a table of contents with \texttt{article} class.
 
\section{A Subheading}                  % Print a "section" heading
The following sectioning commands are available:
\begin{quote}                           % The following text will be
 part \\                                %    set off and indented.
 chapter \\                             % \\ forces a new line
 section \\ 
 subsection \\ 
 subsubsection \\ 
 paragraph \\ 
 subparagraph 
\end{quote}                             % End of indented text
But note that---unlike the \texttt{book} and \texttt{report} classes---the
\texttt{article} class does not have a ``chapter" command.
 
\end{document}                          % The required last line
